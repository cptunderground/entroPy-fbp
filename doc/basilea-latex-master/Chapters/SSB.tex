\chapter{Related Work}
In this chapter we will see some of the key features which were taken account of to realise the
feed bundle protocol. Befor taking a closer at the baseline for the protocol, which consists of
parts from the Secure Scuttlebutt technologie and the Remote Procedure Call Protocol, we
jump shortly into the blockchain and its properties, since by this very moment, everybody
who reads this thesis will heard about.
\section{Blockchain}
The blockchain is well known as the foundation of the bitcoin. It has received an extensive
attention in the recent years (\citet{8029379}). But what is the thing that makes the blockchain so impressiv
and desired? The blockchain is described by \citet{8029379} as an immutable ledger which allows transactions to take place in a decentralised manner. Exactly these key properties we also find
in Secure Scuttlebutt.
\section{Secure Scuttlebutt}
Having the blockchain as a foundation and rather well know by the broad mass for an
append-only log makes the jump into the the universe of Secure Scuttlebutt much easier.
Secure Scuttlebutt (SSB) is a novel peer-to-peer event-sharing protocol and architecture for
social apps.\footnote{Tschudin Paper} Aim of this section is to give a very high level overview about SSB, its ideas
and properties, since the they are not quiet easy to understand.

\subsection{Append-Only Log}

\subsection{Onboarding}
\section{Remote Procedure Call}
Remote procedure calls, as the name implicates, are based on procedure calls but extended
to provide for transfer of control and data across a communication network. There are
two participants in the simplest manner, caller and callee. The caller wants to invoke a
procedure with given parameters. The callee is the instance, which actually proceeds with
the data and returns the result of that specific request. If an RPC is invoked, the the caller’s
environement is supended, all the information needed for the call transmitted through the
network and received by the callee, where the actual procedure is executed with these exact
parameters. The benefit of such an RPC-protocol is that the interfaces are designed in a
way, that third parties only write the procedures and call exactly these procedures in the
callers environement.\citet{birrell1984implementing} This leads to a very promissing way to have a basic version of such
an RPC-protocol for this thesis. Since it allows to invoke in the callers environement but
are atually performend by a callee which returns the result back to the caller.\citet{birrell1984implementing}
