\chapter{Related Work}
In this chapter we examine some of the key features which were taken into account when creating the Feed Bundle Protocol. Before taking a closer at the baseline for the protocol, which consists of parts from the Secure Scuttlebutt technology [\citenum{ssbc}] and the Remote Procedure Call Protocol, we will quickly jump into the blockchain and its properties, everybody who reads this thesis will have heard about it by now.
\section{Blockchain}
The blockchain is well known as the foundation of the bitcoin. It has received an extensive attention in recent years (\citet{8029379}). But what is it that makes the blockchain so impressive and desired? The blockchain is described by \citet{8029379} as an immutable ledger which allows transactions to take place in a decentralised manner. Exactly these key properties we also find in Secure Scuttlebutt.
\section{Secure Scuttlebutt}
Having a rather well known append-only log like the blockchain as a foundation and rather well know by the broad mass makes the jump into the the universe of Secure Scuttlebutt much easier. Secure Scuttlebutt is a novel peer-to-peer event-sharing protocol and architecture for social apps.\footnote{Tschudin Paper} The aim of this section is to give a very high level overview about SSB, its ideas
and properties, since the they are not quite easy to understand.

\subsection{Append-Only Log}

\subsection{Onboarding}
Onboarding in Secure Scuttlebutt differs from the

\citenum{pubs}
\section{Remote Procedure Call}
Remote procedure calls, as the name implies, are based on procedure calls but extended
to provide for transfer of control and data across a communication network. There are
two participants in the simplest manner, the caller and the callee. The caller wants to invoke a
procedure with given parameters. The callee is the instance, which actually proceeds with
the data and returns the result of that specific request. If an RPC is invoked, the the caller’s
environment is supended and all the information needed for the call transmitted is through the
network and received by the callee, where the actual procedure is executed with these exact
parameters. The benefit of such an RPC-protocol is that the interfaces are designed in a
way that third parties only write the procedures and call exactly these procedures in the
callers environment (\citet{birrell1984implementing}) This leads to a very promising basic version of such
an RPC-protocol for this thesis, since it allows the caller to invoke the procedure in the caller's own  environment, but the procedure is actually performend by a callee, which returns the result back to the caller.\citet{birrell1984implementing}
