% !TEX root = ../Thesis.tex
\chapter{Introduction}
The world is constantly changing and so is the internet. At this very moment, a revolution in networking research is taking shape. This movement is leading away from well-known, proven practices and measurements of the centralized web and strives for novelty: distribution. The revolution wants to put an end to the sole reing of million-dollar companies over gathered data from its users. The direction is away from centralised servers and classical Client-Server systems and moving towards routing into a new peer-to-peer-driven, distributed and decentralized web. Secure Scuttlebutt is exactly one of these new developments, which captivate with refreshingly different approaches to solving common networking problems. Yet they are still in development and have a future that is anything but sure. 
\section{Secure Scuttlebutt}
Secure Scuttlebutt (SSB), invented and created by Dominic Tarr in 2014 [\citenum{Tarr}], is a gossip peer-to-peer communication protocol [\citenum{tarr2019secure}]. The term scuttlebutt is slang for "water-cooler" gossip used by sailors and boatsmen. Coincidentally his motivation to develop such a protocol was an unreliable internet connection on his sailboat and the result was his own offline-friendly secure gossip protocol for social networking. \footnote{P2P-Basel \citenum{p2pbasel}}\\

Differing from other technologies, Secure Scuttlebutt does not offer a self-explanatory out of the box onboarding principle. In other software, the user typically receives suggestions for content (e.g. Instagram) or connectivity and management are built into the software (e.g. default gateway DHCP). In SSB, the user has to connect manually to a pub via an invite code, which they must obtain on a channel other than SSB [\citenum{ssbc}].
\section{Motivation}
However, it is problematic for new users to connect to the SSB world, hence a very interesting and intriguing problem has presented itself. SSB is a promising, innovative, new technologie that has a great deal of potential. At the moment, it is still in an experimental state and used primarily in pilot projects where the technology is connected to existing domains (social network, git, databases etc. [\citenum{ssbnzhb}]). This thesis would like to explore its potential in a more commercial manner and environment. 

\section{Goal}
This thesis explores the role of intermediary service providers (ISP) which sell connectivity e.g. to Google or Facebook, through a prototype implementation of a Feed Bundling Protocol (FBP). It is based on SSB, but also differs in several concepts. Introducing these intermediary participants, where you are connected on start up, will make the onboarding easier, since they will hold all the information to create new connections. In plain English: \textit{It’s a guy who knows another guy who can help.} With feed-pairs, which are described later in this thesis, the ID-centric information gathered into one single feed from SSB is split into parts. This results in less data in each dialog between two participants and allows bundling. 

\section{Outline}
First, a more detailed description of SSB, with a focus on the concepts and problems connected to the Feed Bundle Protocol, will be given. Then, the newly created and adapted concepts, as well as the architectural idea of the FBP with respect to SSB, will be presented. Subsequently We will take a closer look at the implemented code and how it has been solved. This evaluation covers previously solved issues, as well as newly generated problems with the approach and how they might possibly be solved. Finally, the thesis will present a conclusion and highlight future challenges discovered during the process. 

