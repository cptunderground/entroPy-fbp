\chapter{Conclusion and Future Work}
\section{Conclusion}
The goal of this project was to introduce new intermediary service providers and replace the ID-centric append-only log from Secure Scuttlebutt Protocol [\citenum{ssbc}] with a feed-pairs, which hold information of the dialog of two identities in the Feed Bundle Protocol. This extension or modification allows a much easier onboarding experience, since the client is indirectly connected to all the ISP’s servers after signing a contract with an ISP. With the new introduce-detruce architecture, clients can connect and disconnect to new servers in a simple manner. Within this process, new feed pairs are created which bundle all the information for that specific connection. To ease the load on the wire and the process of replicating every feed through the ISP directly to the server, requests are multiplexed into a single feed pair between the ISP and the Server. Whether this approach is more promising than the ID-centric architecture of Secure Scuttlebutt can not yet be determined. Since this system is so new and has been developed completely from scratch, there are many ways to improve it, one of which is to use the feeds properties primarily to define the state of doneness inside a feed and its single log entries. There are still many avenues to be explored and important key features to be added in order to produce a reliable Client-ISP-Server Network, some of which are discussed in the next section.

\section{Future Work}
Apart from improving the general system, we have only examined the connection between a single ISP with a single connectivity node with a random number of clients and servers connected. In the real world, however, this is not the case. There are many ISPs on the market which have connectivity stations all over their respective countries. Therefore internet service providers (ISPs) and internet connectivity providers (ICPs) can be separated. Contracts between two ISPs would also be conceivable. In the process of creating the simplified version of the feed bundle protocol, we always kept the big picture in the foreground and decisions were made keeping this in mind. Additionally, a way to combine log entries in the bundling process must be considered to effectively reduce the replication work.

\subsection{Combination of Log Entries}
Seen in the concept and the implementation, only a single new log entry is multiplexed into the ISP-server feed pair, resulting in a replication after each request. A different approach can be made. We can combine log entries in the multiplexing system, which means that instead of only one log entry, a defined number of new log entries or log entries generated over an elapsed time will be multiplexed to the server. This allows for more efficient replication, since the entire feed gets replicated to the peer every time. Deriving from this the multiplexing feed-pair can be split into an arbitrary number of sub-feed pairs, linked to the priority of the request.
\subsection{ISPs and ICPs}
As mentioned previously, the ISP was always a single node, with contracts to clients and servers. However, we could also look at the ISP as a company with a network of ICPs where the physical connection between the servers and clients takes place. In simple terms, the client and server were indirectly ’connected’ through the ISP. The initial mind map drawn to lay out the concept of this thesis was a peer-to-peer Internet Connectivity Provider (ICP) network where the ISP-Company distributes the feeds internally between the ICPs. In the real world, there is a contract with the ISP, e.g. Swisscom, and this same ISP has connectivity provider stations or nodes which form a network of ICPs. This means that a client has a connection to ICP342 of Swisscom and the server has a contract with ICP903. But both have a contract with Swisscom, which provides internal replication and bundling of feed-pairs to pass information from ICP342 to ICP903. 
\begin{figure}
    \centering
    \includegraphics[width=0.9\textwidth]{p2p_contracts}
    \caption{A simplified contract network.}
    \label{fig:contract_network}
\end{figure}
In light of this fact, a new challenge emerges. How are the feeds replicated? How will an introduce-request happen? Can a client have several ICPs connected? Either with the same approach as is currently the case, where every ICP node stores a replication of each feed-pair which it routes to the next node, or by appending only the multiplexed log entries to the ICP-ICP feed-pair. Even a hybrid solution can be considered and explored. Additionally, terminating a contract with one ICP should be possible for a client to change the connectivity provider, for example when traveling from Basel to Zurich. New algorithms need to be developed to handle exactly such use cases.

\subsection{Contracts between ISPs}
Yet again, we can take this distribution to the next level where ISPs have contracts with other ISPs. This provides a way to bypass the current requirement that each ISP must have a contract with any server in the FBP-universe with which it wants to deal. This has a very important impact on the system however, since new contracts are generated when the business aspect has not yet been defined.\\

